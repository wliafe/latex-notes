\documentclass[a4paper,12pt,UTF8]{ctexart}
\usepackage{MyStyle}

\begin{document}
    \title{高等数学解题思路}
    \author{王泠风}
    \date{2024 年 10 月 07 日}
    \maketitle

    \section{一元函数微分学}

    \subsection{求导数}

    \subsubsection{求高阶导数}

    \paragraph{归纳法}

    \paragraph{高阶求导公式}

    \paragraph{泰勒公式(2017年9题)}
    将 \(f(x)\) 化为已知的泰勒展开式,再通过比较系数求出 \(f^{(n)}(x_0)\)。

    \section{几何积分}

    \subsection{曲线积分}

    \subsubsection{积分与路径无关}
    \begin{align*}
        \text{积分与路径无关} \Rightarrow 
        \begin{cases}
            Pdx+Qdy,& \text{是} f(x, y) \text{的全微分},\\
            \oint_{L} Pdx+Qdx = 0,& \text{即} \frac{\partial Q}{\partial x} = \frac{\partial P}{\partial y}.
        \end{cases}
    \end{align*}
    

\end{document}